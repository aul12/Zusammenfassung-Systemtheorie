\chapter{Berücksichtigung von Führungs- und Störgrößen}
\section{Problemstellung}
Ausgangspunkte: LTI System mit zusätzlichem Störeingang
\begin{eqnarray}
    \dot{x} &=& Ax + Bu + Gd,\ t>0, x(0)=x_0 \in \mathbb{R}^n \\
    y &=& C x
\end{eqnarray}
Asymptotisches Folgen soll sichergestellt werden:
\begin{equation}
    \lim_{t \to \infty} e_y = \lim_{t \to \infty} y(t) - w(t) \stackrel{!}{=} 0 
\end{equation}

\section{Modellierung von Führungs- und Störgrößen}
Führungs- und Störgrößen als Lösung einer ODE darstellen:
\begin{eqnarray}
    \dot{v} &=& Sv,\ t>0, v(0)=v_0\in \mathbb{R}^{n_v} \\
    d &=& Pv,\ t \geq 0 \\
    w &=& Qv,\ t \geq 0
\end{eqnarray} 
mit $(P,S)$ und $(Q,S)$ beobachtbar. Resultat: Es muss nur der (unbekannte) Anfangszustand
$v_0$ Kompensiert werden, keine Dauerstörung, d.h. Zustandsrückführung ausreichend.


\section{Führungs- und Störgrößenaufschaltung}
Die Zustandsrückführung mit Führungs- und Störgrößenaufschaltung
\begin{equation}
    u = -Kx + (K \Pi + \Gamma) v
\end{equation}
erzielt asymptotisches Folgen, d.h.
\begin{equation}
    \lim_{t \to \infty} e_y(t) = 0,\ \forall x(0) \in \mathbb{R}^n, \forall v(0) \in
        \mathbb{R}^{n_v}
\end{equation}
wenn
\begin{itemize}
    \item $A-BK$ ist Hurwitz Matrix
    \item $\Pi \in \mathbb{R}^{n\times n_v}$ und $\Gamma \in \mathbb{R}^{p \times n_v}$ sind Lösung der \glqq{}regulator equations\grqq{}:
        \begin{eqnarray}
            \Pi S - A \Pi &=& GP + B \Gamma \\
            C \Pi &=& Q
        \end{eqnarray}
\end{itemize}

\subsection{Starke Übertragungsblockierung}
Das Eingangssignal $u(t) = u_0 e^{\eta t}, \eta \notin \sigma(A)$, wird für $x(0)=c_p$
nicht zum Ausgang übertragen, bzw. stark übertragungsblockiert, falls:
\begin{equation}
    \begin{pmatrix}
        A - \eta I & B \\
        C & 0
    \end{pmatrix}
    \begin{pmatrix} c_p \\ u_0 \end{pmatrix} = 0
\end{equation}

\subsection{Invariante Nullstellen}
Die Lösungen $\eta$ von
\begin{equation}
    \det P(s) = 0
\end{equation}
mit der Rosenbrock-Matrix
\begin{equation}
    P(s) = 
    \begin{pmatrix}
        A - s I & B \\
        C & 0
    \end{pmatrix}
\end{equation}

heißen invariante Nullstellen der Strecke $(C, A, B)$ mit der Zustandsnullstellenrichtung
$c_p$ und der Eingangsnullstellenrichtung $u_0$.

Hinweise:
\begin{itemize}
    \item $\det F(s)=0$ kann nicht zur Berechnung der invarianten Nullstellen genutzt werden
    \item Invariante Nullstellen sind Systemgrößen des Zustandsmodells, nicht des
        Übertragungsverhaltens.
    \item Können nicht durch Zustandsrückführung verschoben werden
\end{itemize}

\subsection{Lösbarkeit der Regulator Equations}
Die \glqq{}regulator equations\grqq{} haben für $\dim y = \dim u$ eine Eindeutige Lösung
$(\Pi, \Gamma)$, wenn keine invarianten Nullstellen von $(C, A, B)$ mit einem
Eigenwert von $S$ übereinstimmt.

\subsection{Diskussion}
Vorteile:
\begin{itemize}
    \item Regelgröße $y$ muss nicht gemessen werden
    \item Keine Verringerung der Stabilitätsreserve der Regelung
\end{itemize}

Nachteile:
\begin{itemize}
    \item Asymptotische Störkompensation vom Eingriffsort abhängig
    \item Nicht robust (bei Modellunbestimmtheiten)
\end{itemize}
 
\section{Internes-Modell-Prinzip}
Annahme: $w, d$ wieder über Signalmodell $S$, $P$, $Q$ können aber unbekannt sein.

Signalmodell:
\begin{equation}
    \dot{\bar{v}}_i = S \bar{v}_i + b_{y_i} (y_i - w_i)
\end{equation}
Anregung mit Eigenschwingungen, also Aufklingen von $\bar{v}_i$. Stabilisierung
des erweiterten Systems (System + Signalmodell).

\subsection{Davisonansatz}
Zusammenfassung der Signalmodelle (ohne Führungsgröße):
\begin{equation}
    \dot{\bar{v}} = S^* \bar{v} + B_y y
\end{equation}
mit ($b_y$ jeweils Steuerbar)
\begin{eqnarray}
    \bar{v} &=& \begin{pmatrix}
            \bar{v}_1 \\ \vdots \\ \bar{v}_p
        \end{pmatrix} \\
    S^* &=& \begin{pmatrix}
            S \\
             & \ddots \\
             & & S
        \end{pmatrix} \\
    B_y &=& \begin{pmatrix}
            b_{y_1} \\
             & \ddots \\
             & & b_{y_p}
        \end{pmatrix}
\end{eqnarray}

erweitertes System:
\begin{eqnarray}
    \begin{pmatrix} \dot{\bar{v}} \\ \dot{x} \end{pmatrix} &=&
        \begin{pmatrix} S^* & B_y C \\ 0 & A \end{pmatrix}
        \begin{pmatrix} \bar{v} \\ x \end{pmatrix} +
        \begin{pmatrix} 0 \\ B \end{pmatrix} u \\
    y &=& \begin{pmatrix} 0 & C \end{pmatrix}
        \begin{pmatrix} \bar{v} \\ x \end{pmatrix}
\end{eqnarray}

\subsection{Stabilisierbarkeit des erweiterten Systems}
Das erweiterte System
\begin{eqnarray}
    \begin{pmatrix} \dot{\bar{v}} \\ \dot{x} \end{pmatrix} &=&
        \begin{pmatrix} S^* & B_y C \\ 0 & A \end{pmatrix}
        \begin{pmatrix} \bar{v} \\ x \end{pmatrix} +
        \begin{pmatrix} 0 \\ B \end{pmatrix} u \\
    y &=& \begin{pmatrix} 0 & C \end{pmatrix}
        \begin{pmatrix} \bar{v} \\ x \end{pmatrix}
\end{eqnarray}
ist steuerbar, wenn
\begin{itemize}
    \item $(A, B)$ steuerbar
    \item $(S, b_{y_i}$ ($i \in \{1, \ldots, p\}$) steuerbar
    \item Keine invariante Nullstelle von $(C, A, B)$ stimmt mit einem
        Eigenwert von $S$ überein.
\end{itemize}

Ergebnis: Dynamische Zustandsrückführung für robustes asymptotisches Folgen:
\begin{eqnarray}
    \dot{\bar{v}} &=& S^* \bar{v} + B_y (y-w) \\
    u &=& -K_{\bar{v}} \bar{v} - K_x x
\end{eqnarray}

\subsection{Diskussion}
Vorteile:
\begin{itemize}
    \item Robust
    \item Unabhängig vom Eingriffsort
\end{itemize}

Nachteile:
\begin{itemize}
    \item Reglerdifferenz muss messbar sein
    \item Signalmodell verringert Stabilitätsreserve
    \item Regler hat Eigenwerte auf der Imaginärachse (Windup Vermeidung)
\end{itemize}
