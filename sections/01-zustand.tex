\chapter{Zustandsbeschreibung linearer Mehrgrößensysteme}
\section{Zustandsbeschreibung}
\subsection{Zustandsdarstellung eines nichtlinearen Systems}
Das Anfangswertproblem (AWP)
\begin{equation}
    \dot{x} = f(x, u), t > 0, x(0) = x_0 \in \mathbb{R}^n
\end{equation}
ist die zeitkontinuierliche Zustandsdarstellung $n$-ter Ordnung eines nichtlinearen
Systems für eine nichtlineare Vektorfunktion $f(x(t), u(t)): \mathbb{R}^n \times
\mathbb{R}^p \to \mathbb{R}^n$ mit dem Zustand $x(t) \in \mathbb{R}^n$ und der
Eingangsgröße $u(t) \in \mathbb{R}^p$ ($1 \leq p \leq n$), wenn es für den Anfangswert
$x(0) \in \mathbb{R}^n$ wohlgestellt ist.

\subsection{Wohlgestellheit nach Hadamard}
Es muss gelten:
\begin{itemize}
    \item Lösung des AWP existiert und ist eindeutig
    \item Lösung des AWP hängt stetig vom Anfangswert ab
\end{itemize}

Ansätze zur Überprüfung:
\begin{itemize}
    \item Zustandsdarstellungen die aus der Physik abgeleitet sind, sind in der Regel
        wohlgestellt
    \item Existenz- und Eindeutigkeitssätze aus der Theorie der Differentialgleichungen
\end{itemize}

\subsection{Ausgang}
\subsubsection{Definition}
In der Regelungstechnik sind nur manche Zustandsgrößen von Interesse, daher wird
eine Ausgangs-  oder Messgröße eingeführt:
\begin{equation}
    y = h(x,u)
\end{equation}
mit $y \in \mathbb{R}^m$. Da der Durchgriff nur in wenigen Systemen relevant ist
wird er oft weggelassen.

\subsubsection{Statisches System}
Besitzt ein System keinen Zustand, so wird es statisches System genannt:
\begin{equation}
    y = h(u)
\end{equation}

\subsubsection{Mehrgrößensysteme}
Gilt $p>1$ oder $m>1$, so spricht man von einem Mehrgrößensystem.

\subsection{Lineare Systeme}
Ein System:
\begin{eqnarray}
    \dot{x} &=& f(x, u),\ t>0, x(0)=x_0 \in \mathbb{R}^n \\
    y &=& h(x),\ t\geq 0
\end{eqnarray}
mit der Lösung $x(t) = \varphi(x_0, u(t), t)$ heißt linear, falls für alle Zeitpunkte
$t \geq 0$ und für alle Anfangszustände $x_0$ und Eingänge $u(t)$ für die
Ausgangsgröße $y(u, x_0, t) = h(\varphi(x_0, u(t), t))$ gilt:
\begin{eqnarray}
    y(\alpha_1 u_1 + \alpha_2 u_2, 0, t) &=& \alpha_1 y(u_1, 0, t) + \alpha_2 y(u_2, 0, t) \\
    y(0, \beta_1 x_{0,1} + \beta_2 x_{0,2}, t) &=& \beta_1 y(0, x_{0,1},t) + \beta_2 y(0, x_{0,2}, t) \\
    y(u, x_0, t) &=& y(0, x_0, t) + y(u, 0, t)
\end{eqnarray}
für $\alpha_1, \alpha_2, \beta_1, \beta_2 \in \mathbb{R}$ beliebig.

Hinweise:
\begin{itemize}
    \item Es wird keine Zustandslinearität, nur E/A Linearität gefordert
    \item Die Definition fordert keine Zeitinvarianz
\end{itemize}

\subsection{LTI Systeme}
Ein lineares Zeitinvariantes System kann durch
\begin{eqnarray}
    \dot{x} &=& Ax + Bu,\ t>0, x(0) = x_0 \in \mathbb{R}^n \\
    y &=& C x,\ t \geq 0 
\end{eqnarray}
beschrieben werden. Hierbei gilt: $A \in \mathbb{R}^{n \times n}, B \in \mathbb{R}^{n \times p}, C \in \mathbb{R}^{m \times n}$.

Es wird stets angenommen: $\rank{B} =p$ und $\rank{C}=m$.

\subsection{Ruhelage}
Die konstante Lösung $x_R(t) =x_R$ heißt Ruhelage des nichtlinearen Systems
\begin{equation}
    \dot{x} = f(x,u)
\end{equation}
wenn für $u(t) = u_R$ gilt:
\begin{equation}
    f(x_R, u_R) = 0
\end{equation}

\subsection{Jacobi-Linearisierung}
Eine Funktion $f: D \to \mathbb{R}^n$ sei auf dem Gebiet $D \subseteq \mathbb{R}^n$
zweimal stetig differenzierbar und die Punkte $x_0$ und $x_0 + \Delta x$ liegen mitsamt
ihrer Verbindungsstrecke in $D$. Dann gilt:
\begin{equation}
    f(x) = f(x_0) + \frac{\partial f}{\partial x}(x_0) (x-x_0) + \mathcal{O}^{[1+]}
        (x-x_0)
\end{equation}
hierbei erfüllt das Restglied:
\begin{equation}
    \lim_{x\to x_0} \frac{\mathcal{O}^{[1+]}(x-x_0)}{\abs{x-x_0}} = 0
\end{equation}

\section{Lösung der Zustandsgleichung}
\subsection{Transitionsmatrix}
Das unerregte System $\dot{x} = Ax$ wird gelöst durch:
\begin{equation}
    x(t) = \Phi(t) x_0
\end{equation}
mit der Transistionsmatrix
\begin{equation}
    \Phi(t) = e^{At}
\end{equation}

\subsection{Eigenschaften der Transitionsmatrix}
Die Transitionsmatrix $\Phi(t): \mathbb{R}^+ \to \mathbb{R}^{n \times n}$ besitzt folgende
Eigenschaften:
\begin{itemize}
    \item Anfangswert:
        \begin{equation}
            \Phi(0) = I
        \end{equation}
    \item Produkteigenschaft:
        \begin{equation}
            \Phi(t + \tau) = \Phi(t) \Phi(\tau)
        \end{equation}
    \item Invertierbarkeit:
        \begin{equation}
            \Phi^{-1}(t) = \Phi(-t) 
        \end{equation}
    \item Differenzierbarkeit:
        \begin{equation}
            \frac{\text{d}}{\text{d}t} \Phi(t) = A \Phi(t)
        \end{equation}
\end{itemize}
Damit bildet die Transitionsmatrix eine Gruppe.

\subsection{Lösung der Zustandsdifferentialgleichung}
Die ZDGL eine LTI-Systems:
\begin{equation}
    \dot{x} = Ax + Bu,\ t>0, x(0) = x_0 \in \mathbb{R}^n
\end{equation}
besitzt die eindeutige Lösung
\begin{equation}
    x(t) = e^{A t} x(0) + \int_0^t e^{A (t-\tau)} B u(\tau) \text{d}\tau
\end{equation}

\subsection{Darstellung über Eigenvektoren}
$w_i$ sind die Linkseigenvektoren, $v_i$ die Rechtseigenvektoren ($W=V^{-1}$) von $A$.

Dann gilt:
\begin{equation}
    x(t) = \sum_{i=1}^n v_i \T{w_i} e^{\lambda_i t} x(0)
        = \sum_{i=1}^n (\T{w_i} x(0)) e^{\lambda_i t} v_i
\end{equation}

Interpretation:
\begin{itemize}
    \item Eigenwerte legen die Form der Eigenschwingung fest
    \item Eigenwerte bestimmen Verteilung der Schwingung auf $x$
    \item Anfangswerte bestimmen die Anregung der Eigenbewegungen
\end{itemize}

\section{Zustandsstabilität}
\subsection{Asymptotische Stabilität}
Ein lineares System heißt asymptotisch stabil, wenn die Zustandstrajektorie $x(t)$ für
$t \to \infty$ gegen $x_R=0$ strebt, und zwar $\forall x_0 \in \mathbb{R}^n$.

\subsection{Eigenwertkriterium}
Ein lineares System ist genau dann asymptotisch stabil, falls $\Re(\lambda_i)<0,
\forall i \in \{1, \ldots, n\}$.

\subsection{Exponentielle Stabilität}
Ein lineares, asymptotisch stabiles System ist stets auch exponentiell stabil,
d.h.:
\begin{equation}
    \norm{x(t)} \leq M e^{-\alpha t} \norm{x_0},\ t \geq 0, \forall x(0) \in \mathbb{R}^n
\end{equation}
für $M\geq 1$ mit der Stabilitätsreserve $\alpha=-\max_{\lambda \in \sigma(A)}(\Re(\lambda))
$

\section{Steuerbarkeit}
\subsection{Definition}
Ein lineares System
\begin{equation}
    \dot{x} = Ax+Bu,\ t>0, x(0)=x_0 \in \mathbb{R}^n
\end{equation}
heißt steuerbar, wenn es in endlicher Zeit $T$ von einem beliebigen Anfangszustand $x(0)
\in \mathbb{R}^n$ durch eine Eingangsgröße $u(t), 0 \leq t \leq T$ in jeden beliebigen
Endzustand $x(T) \in \mathbb{R}^n$ überführt werden kann.

\subsection{Cayley-Hamilton}
Jede Matrix $A \in \mathbb{R}^{n \times n}$ genügt ihrer characteristischen Gleichung
\begin{equation}
    \det(\lambda I - A) = 0
\end{equation}
d.h. $A^n$ ist linear abhängig von $A^i, i \in \{0, \ldots, n-1\}$.

\subsection{Kalmansches Steuerbarkeitskriterium}
Das System
\begin{equation}
    \dot{x} = Ax + Bu,\ t>0, x(0)=x_0\in \mathbb{R}^n
\end{equation}
ist genau dann steuerbar, wenn die Steuerbarkeitsmatrix
\begin{equation}
    Q_s = \begin{pmatrix}
        B & AB & A^2 B & \ldots & A^{n-1} B
    \end{pmatrix}
\end{equation}
vollen Rang $n$ hat.

\section{Beobachtbarkeit}
\subsection{Definition}
Ein lineares System
\begin{eqnarray}
    \dot{x} &=& Ax + Bu,\ t>0, x(0)=x_0\in \mathbb{R}^n \\
    y &=& C x,\ t \geq 0
\end{eqnarray}
ist beobachtbar, wenn ein beliebiger Anfangszustand $x(0)$ aus Kenntnis des Ein- und
Ausgangs $u(t)$ und $y(t)$ auf einem Intervall $0\leq t \leq T$ in endlicher Zeit
bestimmt werden kann.

\subsection{Kalmansches Beobachtbarkeitskriterium}
Das System
\begin{eqnarray}
    \sysm \\
    \sysom \\
\end{eqnarray}
ist genau dann beobachtbar, wenn die Beobachtbarkeitsmatrix
\begin{equation}
    Q_B = \begin{pmatrix}
        C \\ C A \\ \vdots \\ C A^{n-1}
    \end{pmatrix}
\end{equation}
vollen Rang $n$ hat, d.h. $\rank{Q_B} = n$.

\section{Ein-/Ausgangsverhalten}
\subsection{Übertragungsmatrix}
Die Übertragungsmatrix des LTI Systems
\begin{eqnarray}
    \sysm \\
    \sysom
\end{eqnarray}
ist die analytische Fortsetzung von
\begin{equation}
    F(s) = C{(sI-A)}^{-1}B
\end{equation}

\subsection{Gewichtsmatrix}
Die Gewichtsmatrix des LTI-Systems
\syo
ist durch
\begin{equation}
    G(t) = C e^{At} B,\ t \geq 0
\end{equation}
gegeben.

\subsection{BIBO-Stabilität}
Das LTI-System
\syo
heißt BIBO-Stabil, wenn für beliebige beschränkte Eingangssignale, d.h.
\begin{equation}
    \norm{u(t)} \leq u_\text{max} < \infty,\ t \geq 0
\end{equation}
auch das Ausgangssignal beschränkt bleibt, d.h.:
\begin{equation}
    \norm{y(t)} \leq y_\text{max} < \infty,\ t \geq 0
\end{equation}

\subsection{Kriterium für BIBO-Stabilität}
Das LTI-System
\syo
ist BIBO-stabil genau dann, wenn alle Elemente $G_{ij}(t)$ der Gewichtsmatrix $G(t)$
absolut integrierbar sind, d.h.
\begin{equation}
    \int_0^\infty \abs{G_{ij}(\tau)} \text{d}\tau < \infty,\ 
    i \in \{1, \ldots, m\}, j \in \{1, \ldots, p\}
\end{equation}
oder die Übertragungsfunktion $F(s)$ keine Pole in der abgeschlossenen rechten komplexen
Halbebene besitzt.

Bemerkung: Aus Zustandstabilität folgt Übertragungsstabilität, nicht jedoch in die andere
Richtung.
