\chapter{Stabilisierung linearer Systeme im Zustandsraum}
\section{Stabilisierbarkeit}
\subsection{Kalman Kriterium}
Wenn das System steuerbar ist, so sind alle Eigenwerte frei vorgebbar.

\subsection{Hautus Kriterium}
Ein Eigenwert $\lambda$ ist durch eine Zustandsrückführung verschiebbar, falls
\begin{equation}
    \rank \begin{pmatrix} A-\lambda I & B \end{pmatrix} = n
\end{equation}
gilt. In diesem Fall bezeichnet man den Eigenwert $\lambda$ als steuerbar, sind alle Eigenwerte steuerbar,
so ist das System steuerbar.

\subsection{Stabilisierungbarkeitskriterium}
Ein lineares System ist genau dann stabilisierbar, falls alle instabilen Eigenwerte steuerbar sind.

\section{Eigenwertvorgabe}
\subsection{Zustandsreglerformel der Vollständigen Modalen Synthese}
Mit $\tilde{\lambda}_i$ den neuen Eigenwerten und $p_i$ den Parametervektoren gilt:
\begin{eqnarray}
    \tilde{v}_i &=& {(A-\tilde{\lambda}_i I)}^{-1} B p_i,\ i \in \{1, \ldots, n\} \\
    K &=& \begin{pmatrix} p_1 & \ldots & p_n \end{pmatrix}
            {\begin{pmatrix}\tilde{v}_1 & \ldots \tilde{v}_n \end{pmatrix}}^{-1}
\end{eqnarray}

Bemerkungen:
\begin{itemize}
    \item Für reeles $K$ müssen Eigenwert konjugiert komplex vorgegeben werden
    \item Kein Regelungseigenwert darf mit Streckeneigenwert überein stimmen
    \item $\tilde{\lambda}_i$ und $p_i$ müssen so gewählt werden, das $\tilde{v}_i$
        linear unabhängig
    \item Parametervektoren beeinhalten $n(p-1)$ Fhg, im Mehrgrößenfall
        auch Eigenvektorvorgabe
\end{itemize}

\section{Quadratisch optimaler Regler: Riccati-Regler}
\subsection{Riccati-Regler}
Gegeben sei das quadratische Gütemaß
\begin{equation}
    J = \frac{1}{2} \int_0^\infty \T{x(\tau)} Q x(\tau)
        + \T{u(\tau)} R u(\tau) \text{d} \tau
\end{equation}
mit $Q>0$ und $R>0$. Dann ist die optimale Zustandsrückführung, der sogenannte Riccati-
Regler durch
\begin{equation}
    u = -Kx
\end{equation}
mit
\begin{equation}
    K = R^{-1} \T{B} P
\end{equation}
gegeben, welcher $J$ für das steuerbare LTI-System
\sy
minimiert und exponentiell stabilisiert. Darin ist $P>0$ die eindeutige Lösung der
algebraischen Riccati-Gleichung:
\begin{equation}
    PA+\T{A}P-PBR^{-1}\T{B}P+Q=0
\end{equation}

Bemerkung: es reicht auch $Q>0$ dann muss aber für ein $Q_0$ mit $Q=\T{Q_0}Q_0$ (Cholesky-
Zerlegung) die Beobachtbarkeit von $(Q_0, A)$ gelten.
