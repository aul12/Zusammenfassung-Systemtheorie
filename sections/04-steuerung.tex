\chapter{Steuerungsentwurf im Zustandsraum}
\section{Modellgestütze Vorsteuerung}
\subsection{Führungsgrößenaufschaltung}
\subsubsection{Definition Exaktes Folge}
Die Regelgröße folgt dem Sollverlauf $y_s$ exakt, wenn
\begin{equation}
    y(t) = y_s(t),\ \forall t \geq 0
\end{equation}
bzw. $y \equiv y_s$.

\subsubsection{Definition Asymptotisches Folgen}
Die Regelgröße folgt der Führungsgröße $w$ asymptotisch, wenn
\begin{equation}
    \lim_{t \to \infty} y(t) - w(t) = 0
\end{equation}
dabei soll $y$ den Verlauf von $w$ schnell und gut gedämpft annehmen.

Ansatz: Vorgabe des Folgeverhaltens durch Festlegung des Führungsübertragungsverhaltens:
\begin{equation}
    y_s(s) = F_w(s) w(s) \stackrel{!}{=} y(s)
\end{equation}

\subsubsection{Führungsgrößenaufschaltung}
Ansatz:
\begin{equation}
    u = -K x + M w
\end{equation}
Bestimmung $K$ über Eigenwertvorgabe oder Ricatti Regler.

Vorfilter $M$ für stationäre Genauigkeit:
\begin{equation}
    M = {\left(C{(-A+BK)}^{-1}B\right)}^{-1}
\end{equation}

Voraussetzungen für Existenz:
\begin{itemize}
    \item Kein Regelungseigenwert bei Null
    \item Keine invariante Nullstelle der Strecke bei $0$
\end{itemize}

\subsection{Ein-/Ausgangsentkopplung}
\subsubsection{Definition Differenzgrad}
Die Ausgangsgröße $y_{s,i}$ hat Differenzgrad $r_i$, wenn:
\begin{eqnarray}
    \T{c_i} A^k B &=& \T{0},\ k \in \{0, 1, \ldots, r_i-2\} \\ 
    \T{c_i} A^{r_i-1} B &\neq& \T{0}
\end{eqnarray}
Die Summe $r = \sum_i r_i \leq n$ heißt Differenzgrad des Systems.

\subsubsection{Entkoppelbarkeitskriterium}
Ein LTI-System ist ein-/ausgangsentkoppelbar genau dann wenn $\det D^* \neq 0$ mit:
\begin{equation}
    D^* = \begin{pmatrix}
            \T{c_1} A^{r_1-1} B \\
            \vdots \\
            \T{c_p} A^{r_p-1} B \\
        \end{pmatrix}
\end{equation}

dann können die $p$ Integrierketten entkoppelt werden:
\begin{equation}
    u_s = {(D^*)}^{-1} (-A^* x_s + \bar{u}_s)
\end{equation}

\subsubsection{Entkopplungsregler nach Falb-Wolovich}
Der Entkopplungsregler 
\begin{equation}
    u_s = - {(D^*)}^{-1} \begin{pmatrix}
            \T{k_1^*} \\ \vdots \\ \T{k_m^*}
        \end{pmatrix} +
        {(D^*)}^{-1} \begin{pmatrix}
            \tilde{a}_0^1 \\
             & \ddots \\
             & & \tilde{a}_0^p
        \end{pmatrix} w
\end{equation}
mit
\begin{equation}
    D^* = \begin{pmatrix}
            \T{c_1} A^{r_1-1} B \\
            \vdots \\
            \T{c_p} A^{r_p-1} B \\
        \end{pmatrix}
\end{equation}
und
\begin{equation}
    \T{k^*_i} = \T{c_i} \left(A^{r_i} + \sum_{j=0}^{r_i-1} \tilde{a}_j^i A^j \right)
\end{equation}

bewirkt die Ein-/Ausgangsentkopplung des vorgesteuerten Führungsverhaltens, d.h.:
\begin{equation}
    F_w(s) = \begin{pmatrix}
        \frac{\tilde{a}_0^1}{s^{r_1} + \tilde{a}^1_{r_1-1} s^{r_1-1} + \ldots + \tilde{a}_0^1} \\
        & \ddots \\
        & & \frac{\tilde{a}_0^p}{s^{r_p} + \tilde{a}^p_{r_p-1} s^{r_p-1} + \ldots + \tilde{a}_0^p} 
    \end{pmatrix}
\end{equation}

Stabilität:

$n-r$ Eigenwerte $\tilde{\lambda}_i$ ($i\in\{r+1, \ldots, n\}$) werden mit invarianten
Nullstellen kompensiert.

\subsubsection{Hinreichendes Stabilitätskriterium für stabile Entkopplung}
Die Entkopplungsregelung ist exponentiell stabil, wenn die Strecke minimalphasig ist,
d.h. wenn für alle invarianten Nullstellen $\eta_i$ ($i\in \{1, \ldots, n-r\}$)
\begin{equation}
    \Re(\eta_i) < 0
\end{equation}
gilt.

\subsubsection{Entkoppelbarkeitskriterium}
Ein LTI-System $n$-ter Ordnung mit Differenzgrad $r$ ist ein-/ausgangsentkoppelbar genau
dann, wenn für seine Rosenbrock-Matrix $P(s)$ gilt:
\begin{equation}
    \deg \det P(s) = n-r
\end{equation}

\subsubsection{Dynamische Erweiterung}
Idee: Weitere Integrierer einbauen:
\begin{eqnarray}
    \dot{x} &=& Ax + BG_1 \xi + B G_2 \bar{u}_2 \\
    \dot{\xi} &=& \bar{u}_1 \\
    C &=& Cx
\end{eqnarray}

Bestimmung von $n_\xi$ und $G$ für die dynamische Erweiterung zur Erzielung von
$\tilde{r} = r+p-\gamma > r$:
\begin{itemize}
    \item Ordnung der dynamischen Erweiterung:
        \begin{equation}
            n_\xi = p - \mu - \gamma
        \end{equation}
        mit
        \begin{equation}
            \mu = p - \rank D^*
        \end{equation}
        und einer Matrix $Q \in \mathbb{R}^{\mu \times p}$, $\rank Q = \mu$ mit
        \begin{equation}
            Q D^* = 0
        \end{equation}
        $\gamma$ bezeichnet die $0$-Spalten in $Q$.
    \item Berechnung von $G=\begin{pmatrix}G_L & G_R\end{pmatrix}$ ($G_L$ enthält $p-\mu$ Spalten,
        $G_R$ $\mu$ Spalten) aus:
        \begin{equation}
            \begin{pmatrix} 0 & I \end{pmatrix} D^* G_L = I
        \end{equation}
        (die linke Matrix besteht aus $\mu$-Nullspalten und einer Identitätsmatrix
        der Größe $p-\mu$) und
        \begin{equation}
            D^* G_R = 0
        \end{equation}
        mit $\rank G_R = \mu$.
\end{itemize} 

\subsubsection{Dynamischer Erweiterungsalgorithmus}
Erweitern bis Erweitertes System entkoppelbar:

Abbruchbedingung: 
\begin{equation}
    \delta_{i+1} = \deg \det P(s)
\end{equation}
mit
\begin{equation}
    \delta_{i+1} = \delta_i - (p - \rank D^*_i)
\end{equation}
und
\begin{equation}
    \delta_0 = n - r
\end{equation}

Falls Strecke minimalphasig, terminiert Algorithmus nach endlich vielen Schritten.

\subsection{Vorsteuerung des Störverhaltens}
Ausgangspunkt: Strecke mit messbarer Störung und bekannter Störeingangsmatrix,
Vorsteuerung für asymptotische Störkompensation.

Zustandsrückführung mit Störgrößenaufschaltung im Modellregelkreis:
\begin{equation}
    u_s = -K x_s + M_d d
\end{equation}
$K$ mit Eigenwertvorgabe oder Riccati-Regler, $M_d$ für asymptotische Störkompensation:
\begin{equation}
    M_d = -{\left(C{(-A+BK)}^{-1} B\right)}^{-1} C {(-A + BK)}^{-1}G
\end{equation}
Vorraussetzungen:
\begin{itemize}
    \item Kein Regelungseigenwert bei $0$
    \item Keine Invariante Nullstelle der Strecke bei $0$
\end{itemize}

\section{Inversionsbasierte Ausgangsfolge}
\subsection{Ein-/Ausgangsnormalform}
\begin{enumerate}
    \item Zustandsgrößen $\xi_j^i$ ($j\in\{1, \ldots, r_i\}$) der Ein-/Ausgansdynamik
        zum Ausgang $y_i$:
        \begin{eqnarray}
            \xi_1^i &=& y_i = \T{c_i} x \\
            \xi_2^i &=& \dot{y_i} = \T{c_i} A x \\
            &\vdots& \\
            \xi_{r_i}^i &=& y_i^{(r_i-1)} = \T{c_i} A^{r_i-1} x
        \end{eqnarray}
        für $i \in \{1, \ldots, p\}$ resultiert in $r$ Zustandsgrößen mit:
        \begin{equation}
            \xi = T_\xi x
        \end{equation}
    \item Zustandsgrößen $\eta_i$ ($i\in\{1, \ldots, n-r\}$) der internen Dynamik mit
        \begin{equation}
            \eta = T_\eta x
        \end{equation}
        finde $T_\eta$, so dass
        \begin{equation}
            \det \begin{pmatrix} T_\xi \\ T_\eta \end{pmatrix} \neq 0
        \end{equation}
        so dass die Koordinatentransformation gültig ist.
    \item Eingangsnormalisierung: Falls $T_\eta B =0$ gilt, so hängt das $\eta$-Teilsystem
        nicht direkt vom Eingang ab.
\end{enumerate}

$\eta$ Teilsystem der Ordnung $n-r$ ist in $y$ unbeobachtbar, d.h. Eigenwerte von
$Q$ stimmen mit invarianten Nullstellen überein.

\subsubsection{Nulldynamik}
Die bei einer Ausgangsnullung $y\equiv 0$ noch im System vorhandene Dynamik wird
Nulldynamik genannt und ist durch:
\begin{equation}
    \dot{\eta} = Q \eta,\ \eta(0) = \eta_0 \in \mathbb{R}^{n-r}
\end{equation}
gegeben.

\subsubsection{Starke Übertragungsblockierung}
Die Eigenschwingungen der Nulldynamik sind die einzigen Eingangssignale, welche stark
übertragungsblockiert werden können.

\subsection{Ausgangsfolge}
Inverses System für $r<n$:

Interne Dynamik (numerisch) integrieren:
\begin{equation}
    \dot{\eta}_s(t) = Q \eta_s(t) + P \psi_\xi(y_s(t), \ldots, y_s^{(r-1)}(t))
\end{equation}
für $t>0$ mit $\eta_s(0) = \eta_{s,0} \in \mathbb{R}^{n-r}$

Vorsteuerung durch Entkopplung und Aufprägung des Führungsverhaltens:
\begin{equation}
        u_s(t) = {(D^*)}^{-1} \left(y_s^{(r)}(t) - P^* \psi_\xi\left(y_s(t), \ldots, y_s^{(r-1)}(t)\right)
        - Q^* \eta_s(t)\right)
\end{equation}

\section{Flachheitsbasierter Arbeitspunktwechsel}
\subsection{Arbeitspunktwechsel}
Bestimmung einer Solltrajektorie $x_s$ und einer Steuerung $u_s$ für
das System so dass Arbeitspunktwechseln in endlicher Zeit vollzogen wird.

Für $r=n$ muss kein 2-Punkt Randwertproblem gelöst sondern rein algebraische
Lösung möglich.

\subsection{Flachheit linearer Systeme}
\subsubsection{Definition}
Ein lineares Mehrgrößensystem $n$-ter Ordnung
\sy
mit $\rank B = p$ heißt flach, wenn es einen fiktiven Ausgang
\begin{equation}
    y_f = \T{\begin{pmatrix} y_{f1} & \ldots & y_{f2} \end{pmatrix}}
\end{equation}
so dass
\begin{itemize}
    \item
        \begin{equation}
            y_f = C_f x
        \end{equation}
    \item Und mit $\kappa = (\kappa_1, \ldots, \kappa_p)$:
        \begin{equation}
            x = \psi_x(y_{f1}, \dot{y_{f1}}, \ldots, y_{f1}^{(\kappa_1-1)},
                 y_{fp}, \dot{y_{fp}}, \ldots, y_{fp}^{(\kappa_p-1)})
        \end{equation}
        (differentielle Parametrierung von $x$)
    \item
        \begin{equation}
            x = \psi_u(y_{f1}, \dot{y_{f1}}, \ldots, y_{f1}^{(\kappa_1)},
                 y_{fp}, \dot{y_{fp}}, \ldots, y_{fp}^{(\kappa_p)})
        \end{equation}
        (differentielle Parametierung von $u$) 
\end{itemize}
gelten. In diesem Fall ist $y_f$ ein flacher Ausgang des Systems.
Die höchsten Zeitableitungen in der differentiellen Parametierung ($\kappa$) werden
Steuerbarkeitsindizes genannt.

Für lineare Systeme gilt:
\begin{equation}
    \text{steuerbar} \Leftrightarrow \text{flach}
\end{equation}

\subsection{Bestimmung flacher Ausgänge}
\subsubsection{Characterisierung flacher Ausgänge (hinreichend)}
Ein Ausgang $y_f$ ist ein flacher Ausgang, wenn
\begin{itemize}
    \item Für seine Differenzgrade $\kappa_i$ die Beziehung $\sum_i \kappa_i = n$ gilt und
    \item das System bezüglich des Ausgangs $y_f$ ein-/ausgangsentkoppelbar ist, d.h.
        $\det D^* \neq 0$.
\end{itemize}

\subsubsection{Steuerbarkeitsindizes}
Ist $\kappa_i$ die kleinste ganze Zahl, so dass der Spaltenvektor $A^{\kappa_i} b_i$
von den links in $Q_S$ gelegenen Spalten linear abhängig ist, dann ist $\kappa_i$ der
Steuerbarkeitsindex zur $i$-ten Eingangsgröße $u_i$ ($i \in \{1, \ldots, p\}$).

Auswahlmatrix bestimmen:
\begin{equation}
    \tilde{Q}_S = \begin{pmatrix}
            b_1 & A b_1 & \cdots & A^{\kappa_1-1}b_1 & \cdots & b_p & \cdots &
                A^{\kappa_p-1} b_p
        \end{pmatrix}
\end{equation}

Ein möglicher flacher Ausgang ist dann gegeben durch:
\begin{equation}
    C_f = \begin{pmatrix} \T{e_{\kappa_1}} \\ \vdots \T{e_{\kappa_1 + \ldots + \kappa_p}} \end{pmatrix}
        {\tilde{Q}_S}^{-1}
\end{equation}
mit $e_i \in \mathbb{R}^n$

\subsection{Flachheitsbasierter Steuerungsentwurf}
Vorgehen:
\begin{enumerate}
    \item Bestimmung von Anfangs- und Endpunkt der Solltrajektorie $y_{fs}$
    \item Planung einer Solltrajektorie mit $u_s \in C \Rightarrow
        y_{fi,s} \in C^{\kappa_i}$
    \item Solltrajektorie für den Zustand und Steuerung bestimmen
\end{enumerate}

\subsection{Arbeitspunktwechsel für die Regelgröße}
Solltrajektorie $y_s$ auf Basis von $y_{fi,s}$ berechnen:
\begin{equation}
    y_s = C \psi_x(y_f, \dot{y_f}, \ldots, y_f^{(\kappa-1)})
        = \psi_y(y_f, \dot{y_f}, \ldots, y_f^{(\kappa-1)})
\end{equation}
Regler mit Sollwert $y_s$ bestimmen.

