\chapter{Steuerungsentwurf im Zustandsraum}
\section{Modellgestütze Vorsteuerung}
\subsection{Führungsgrößenaufschaltung}
\subsubsection{Definition Exaktes Folge}
Die Regelgröße folgt dem Sollverlauf $y_s$ exakt, wenn
\begin{equation}
    y(t) = y_s(t),\ \forall t \geq 0
\end{equation}
bzw. $y \equiv y_s$.

\subsubsection{Definition Asymptotisches Folgen}
Die Regelgröße folgt der Führungsgröße $w$ asymptotisch, wenn
\begin{equation}
    \lim_{t \to \infty} y(t) - w(t) = 0
\end{equation}
dabei soll $y$ den Verlauf von $w$ schnell und gut gedämpft annehmen.

Ansatz: Vorgabe des Folgeverhaltens durch Festlegung des Führungsübertragungsverhaltens:
\begin{equation}
    y_s(s) = F_w(s) w(s) \stackrel{!}{=} y(s)
\end{equation}

\subsubsection{Führungsgrößenaufschaltung}
Ansatz:
\begin{equation}
    u = -K x + M w
\end{equation}
Bestimmung $K$ über Eigenwertvorgabe oder Ricatti Regler.

Vorfilter $M$ für stationäre Genauigkeit:
\begin{equation}
    M = {\left(C{(-A+BK)}^{-1}B\right)}^{-1}
\end{equation}

Voraussetzungen für Existenz:
\begin{itemize}
    \item Kein Regelungseigenwert bei Null
    \item Keine invariante Nullstelle der Strecke bei $0$
\end{itemize}

\subsection{Ein-/Ausgangsentkopplung}
\subsubsection{Definition Differenzgrad}
Die Ausgangsgröße $y_{s,i}$ hat Differenzgrad $r_i$, wenn:
\begin{eqnarray}
    \T{c_i} A^k B &=& \T{0},\ k \in \{0, 1, \ldots, r_i-2\} \\ 
    \T{c_i} A^{r_i-1} B &\neq& \T{0}
\end{eqnarray}
Die Summe $r = \sum_i r_i \leq n$ heißt Differenzgrad des Systems.

\subsubsection{Entkoppelbarkeitskriterium}
Ein LTI-System ist ein-/ausgangsentkoppelbar genau dann wenn $\det D^* \neq 0$ mit:
\begin{equation}
    D^* = \begin{pmatrix}
            \T{c_1} A^{r_1-1} B \\
            \vdots \\
            \T{c_p} A^{r_p-1} B \\
        \end{pmatrix}
\end{equation}

dann können die $p$ Integrierketten entkoppelt werden:
\begin{equation}
    u_s = {(D^*)}^{-1} (-A^* x_s + \bar{u}_s)
\end{equation}

\subsubsection{Entkopplungsregler nach Falb-Wolovich}
Der Entkopplungsregler 
\begin{equation}
    u_s = - {(D^*)}^{-1} \begin{pmatrix}
            \T{k_1^*} \\ \vdots \\ \T{k_m^*}
        \end{pmatrix} +
        {(D^*)}^{-1} \begin{pmatrix}
            \tilde{a}_0^1 \\
             & \ddots \\
             & & \tilde{a}_0^p
        \end{pmatrix} w
\end{equation}
mit
\begin{equation}
    D^* = \begin{pmatrix}
            \T{c_1} A^{r_1-1} B \\
            \vdots \\
            \T{c_p} A^{r_p-1} B \\
        \end{pmatrix}
\end{equation}
und
\begin{equation}
    \T{k^*_i} = \T{c_i} \left(A^{r_i} + \sum_{j=0}^{r_i-1} \tilde{a}_j^i A^j \right)
\end{equation}

bewirkt die Ein-/Ausgangsentkopplung des vorgesteuerten Führungsverhaltens, d.h.:
\begin{equation}
    F_w(s) = \begin{pmatrix}
        \frac{\tilde{a}_0^1}{s^{r_1} + \tilde{a}^1_{r_1-1} s^{r_1-1} + \ldots + \tilde{a}_0^1} \\
        & \ddots \\
        & & \frac{\tilde{a}_0^p}{s^{r_p} + \tilde{a}^p_{r_p-1} s^{r_p-1} + \ldots + \tilde{a}_0^p} 
    \end{pmatrix}
\end{equation}

Stabilität:

$n-r$ Eigenwerte $\tilde{\lambda}_i$ ($i\in\{r+1, \ldots, n\}$) werden mit invarianten
Nullstellen kompensiert.

\subsubsection{Hinreichendes Stabilitätskriterium für stabile Entkopplung}
Die Entkopplungsregelung ist exponentiell stabil, wenn die Strecke minimalphasig ist,
d.h. wenn für alle invarianten Nullstellen $\eta_i$ ($i\in \{1, \ldots, n-r\}$)
\begin{equation}
    \Re(\eta_i) < 0
\end{equation}
gilt.

\subsection{Vorsteuerung des Störverhaltens}

\section{Inversionsbasierte Ausgangsfolge}
\subsection{Ein-/Ausgangsnormalform}

\subsection{Ausgangsfolge}

\section{Flachheitsbasierter Arbeitspunktwechsel}
\subsection{Arbeitspunktwechsel}

\subsection{Flachheit linearer Systeme}

\subsection{Bestimmung flacher Ausgänge}

\subsection{Flachheitsbasierter Steuerungsentwurf}

\subsection{Arbeitspunktwechsel für die Regelgröße}
