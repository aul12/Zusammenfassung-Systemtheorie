\chapter{Zwei-Freiheitsgrade-Regelung}
\section{Struktur der Folgeregelung}
\subsubsection{Online Sollvorgabe}
Komponenten:
\begin{itemize}
    \item Modellgestützte $w$-Aufschaltung
    \item Modellgestützte $d$-Aufschaltung
    \item Ausgangsfolgeregler
\end{itemize}

Vorteile: Führungsverhalten, und Störverhalten bezüglich jeweils $d$ und $r$ kann
separat ausgelegt werden.

\subsubsection{Offline Sollvorgabe}
Komponenten
\begin{itemize}
    \item Streckeninverse für Steuerung
    \item Ausgangsfolgeregler
\end{itemize}

\section{Ausgangsfolgeregler}
\subsection{Beobachterbasierte Zustandsrückführung}
Entwurf der beobachterbasierten Ausgangsrückführung:
\begin{enumerate}
    \item Entwurf einer Zustandsrückführung
        \begin{equation}
            u_R = -K e_x
        \end{equation}
    \item Entwurf eines reduzierten Zustandsbeobachters
    \item Implementierung des Stellgesetz auf basis Beobachterzustände
    \item Resultiert in dynamische Ausgangsrückführung:
        \begin{eqnarray}
            \begin{pmatrix}
                \dot{\hat{e_\xi}} \\ \dot{\bar{v}}
            \end{pmatrix}
            &=&
            \begin{pmatrix}
                F - TBR_\xi & - T B R_v \\
                0 & S^*
            \end{pmatrix}
            \begin{pmatrix}
                \hat{e_\xi} \\ \bar{v}
            \end{pmatrix}
            +
            \begin{pmatrix}
                L-TBR_y \\ B_{e_y}
            \end{pmatrix}
            e_y \\
            u_r &=& -R_y e_y - R_\xi \hat{e_\xi} - R_v \bar{v}
        \end{eqnarray}
\end{enumerate}

\subsection{Beobachterbasierte Ausgangsrückführung}
Ansatz: Stellgesetz auf basis des Reduzierten Beobachters kann weiter
reduziert werden wenn Dimension des reduzierten Beobachters ($n_0$) weiter reduziert wird.

\subsubsection{Ausgangsreglerformel}
Die Matrix
\begin{equation}
    R(\tilde{\lambda}_i, p_i, \bar{\lambda}_j, l_j) = -
        \begin{pmatrix} p_1 & \ldots & p_{m+n_0} \end{pmatrix}
        {\begin{pmatrix} G(\tilde{\lambda}_1) p_1 & \ldots &
        G(\tilde{\lambda}_{m+n_0})p_{m+n_0} \end{pmatrix}}^{-1}
\end{equation}
mit
\begin{equation}
    G(s) = \begin{pmatrix}
            F(s) \\
            \frac{\T{l_1}}{s-\bar{\lambda}_1} \left(F(s) - F(\bar{\lambda}_1)\right) \\
            \vdots \\
            \frac{\T{l_{n_0}}}{s-\bar{\lambda}_{n_0}} \left(F(s) - F(\bar{\lambda}_{n_0})\right)
        \end{pmatrix}
\end{equation}
für $\tilde{\lambda}_i \notin \left(\sigma(A) \cup \sigma(\Lambda)\right)$ und
$\bar{\lambda}_i \notin \sigma(A)$ verleiht dem geregelten Fehlersystem:
\begin{equation}
    \dot{e_x} = (A-BRH) e_x
\end{equation}
die Eigenwerte $\tilde{\lambda}_i$ ($i \in \{1, \ldots, m+n_0\}$) und die zugehörigen 
Parametervektoren $p_i$. Darin stellen die Beobachterreigenwerte $\bar{\lambda}_i$ und
ihre Parametervektoren $\T{l_i}$ weitere FHG dar.

\subsubsection{Reduzierter Ausgangsfolgeregler}
Bei Verwendung eines $e_\xi$-Beobachters der Ordnung $n_0 < n-m$ können mit dem 
Ausgangsfolgeregler $m+n_0<n$ Eigenwerte direkt vorgegeben werden, während die restlichen
$n-(m+n_0)$ Eigenwerte durch eine Parameteroptimierung beeinflussbar sind.

\subsection{Separationsprinzip}
Die Eigenwerte des mittels einer beobachterbasierten Zustandsrückführung oder
Ausgangsrückführung geregelten Fehlersystems sind die Nullstellen von:
\begin{equation}
    \det (\lambda I - \tilde{A}) = \det(\lambda I - F) \det(\lambda I - A + BRH)
\end{equation}
also die Eigenwerte des Beobachters und der Fehlerzustandsrückführung, damit bleiben
die Dynamikvorgaben des getrennten Entwurfs im geschlossenen Regelkreis erhalten.
Folglich existiert ein stabilisierender Ausgangsregler, wenn das Fehlersystem (bzw.
die Strecke) steuer- und beobachtbar ist.

\section{Berücksichtigung von Stellsignalbegrenzungen}
\subsection{Ausgangsfolgeregler mit klassischem Davisonansatz}
\subsubsection{Beobachtertechnik}
Durch Einspeisung des begrenzten Stellsignals in den Beobachter wird die Rückführung
über den Eingang im Begrenzungsfall unterbrochen. Damit tritt bei Begrenzungen nur die
stabile Beobachterdynamik im offenen Kreis auf und das Beobachter-Windup wird vermieden.

\vspace{.5cm}

Hinweis: Stellsignalbegrenzung kann auch explizit vor System gehängt werden, damit
wird Stellsignalbegrenzung des Systems nie erreicht.

\vspace{.5cm}

Vermeidung des Windups der Zustände $\bar{v}$ der angeregten Störmodelle:

Einführung der Begrenzungsgröße:
\begin{equation}
    \Delta u  = u - u_b
\end{equation}
und zusätzliche Rückführung im Störmodell:
\begin{equation}
    \dot{\bar{v}} = S^* \bar{v} + B_{e_y} e_y B_{\Delta u} \Delta u
\end{equation}
dafür muss $S^*-B_{\Delta u}R_v$ Hurwitz Matrix sein, das ist möglich falls $(R_v, S*)$
beobachtbar.

\subsection{Ausgangsfolgeregler mit dualem Davisonansatz}
Beobachtertechnik ausreichend.
