\chapter{Lineare Beobachter}
\section{Zustandsbeobachter}
Ansatz: Regelung für Beobachterfehlerdynamik:
\begin{equation}
    \dot{\hat{x}} = A \hat{x} + Bu + L (y - C \hat{x})
\end{equation}

\subsubsection{Dualitätsprinzip für den Beobachterentwurf}
Das Beobachterstabilisierungsproblem lässt sich auf die Stabilisierung des dualen
Systems:
\begin{eqnarray}
    \dot{x}^* &=& \T{A} x^* + \T{C} u^* \\
    y^* &=& \T{B} x^*
\end{eqnarray}
mittels einer Zustandsrückführung
\begin{equation}
    u^* = -\T{L} x^*
\end{equation}
zurückführen.

\subsubsection{Hautus-Kriterium für Beobachtbarkeit}
Ein Eigenwert $\lambda_i$ ist durch eine Beobachterrückführmatrix verschiebbar genau
dann, wenn
\begin{equation}
    \rank \begin{pmatrix} A - \lambda_i I \\ C \end{pmatrix} = n
\end{equation}
gilt. In diesem Fall bezeichnet man den Eigenwert als beobachtbar. Sind alle Eigenwerte
beobachtbar, dann ist das System beobachtbar.

\subsubsection{Stabilisierbarkeitskriterium für die Beobachterfehlerdynamik}
Die Beobachterfehlerdynamik ist genau dann stabilisierbar, wenn alle Eigenwerte
$\lambda_i$ mit $\Re(\lambda_i) \geq 0$ beobachtbar sind.
    
\subsection{Entwurf durch Eigenwertvorgabe}
\subsubsection{Beobachterformel der Vollständigen Modallen Synthese}
Die Beobachterrückführmatrix:
\begin{equation}
    L = {\begin{pmatrix} \T{\tilde{w}_1} \\ \vdots \\ \T{\tilde{w}_n} \end{pmatrix}}^{-1}
        \begin{pmatrix} \T{\bar{p}_1} \\ \vdots \\ \T{\bar{p}_n} \end{pmatrix}
\end{equation}
mit
\begin{equation}
    \T{\tilde{w}_i} = \T{\bar{p}_i} C {\left(A-\bar{\lambda}_i I\right)}^{-1}
\end{equation}
($i \in \{1, \ldots, n\}$). Verleiht der Beobachterfehlerdynamik
\begin{equation}
    \dot{e_B} = (A-LC) e_B
\end{equation}
die Eigenwerte $\bar{\lambda}_i$ und die zugehörigen Parametervektoren $\bar{p}_i$. Die
Vektoren $\tilde{w}_i$, sind dann die links Eigenvektoren der Fehlerdynamik bzw. des
Beobachters.

Hinweise:
\begin{itemize}
    \item Es gelten die selben Einschränkungen wie bei der VMS
    \item Für Eigenwerte bei $-\infty$ konvergiert der Beobachter gegen einen
        Differenzierer, der die inverse Beobachterabbildung bestimmt.
\end{itemize}

\subsection{Quadratisch optimale Zustandsrekonstruktion: stationäres Kalman-Filter}
\subsubsection{Deterministischer Beobachter}
Die Beobachterrückführmatrix
\begin{equation}
    L = \bar{P}\T{C}\bar{R}^{-1}
\end{equation}
mit $\bar{P}>0$ eindeutige Lösung der algebraischen Riccati-Gleichung:
\begin{equation}
    A\bar{P}+\bar{P}\T{A}-\bar{P}\T{C}{\bar{R}}^{-1}C\bar{P}+\bar{Q}=0
\end{equation}
für $\bar{Q}>0$ und $\bar{R}>0$ führt bei einer beobachtbaren Strecke auf die exponentiell
stabile Beobachterfehlerdynamik
\begin{equation}
    \dot{e_B} = (A-LC)e_B
\end{equation}

Optimalitätsannahmen:
\begin{itemize}
    \item $R$ ist weißes, mittelwertfreies Gaußsches Systemrauschen
    \item $Q$ ist weißes, mittelwertfreies Gaußsches Messrauschen
    \item Mess und Systemrauschen sind unkorreliert
\end{itemize}

\subsubsection{Stationäres Kalman-Filter}
Gegeben sei das quadratische Gütemaß
\begin{equation}
    \bar{J}_\infty = \lim_{t \to \infty} \sum_{i=1}^n E\left\{ {\left(
        x_i(t) - \hat{x}_i(t)\right)}^2\right\}
\end{equation}
dann minimiert der oben gegebene Beobachter dieses bei einem Schätzfehler von $P$.

\section{Reduzierter Beobachter}
Idee: $m$ Zustandsgrößen sind direkt aus dem Ausgang rekonstruierbar, also reicht ein
Zustandsbeobachter der Dimension $n-m$.

\subsection{Beobachterformel für den reduzierten Beobachter}
Die Matrix
\begin{equation}
    T(\bar{\lambda}_i,l_i) = \begin{pmatrix}
        \T{l_1} C {\left(A-\bar{\lambda}_1 I \right)}^{-1} \\
        \vdots \\
        \T{l_{n-m}} C {\left(A-\bar{\lambda}_{n-m} I \right)}^{-1} \end{pmatrix}
\end{equation}
mit
\begin{equation}
    \det \begin{pmatrix} C \\ T \end{pmatrix} \neq 0
\end{equation}
verleiht der Beobachterfehlerdynamik $\dot{\xi}-\dot{\hat{\xi}}=F(\xi-\hat{\xi})$
mit $F=\diag(\bar{\lambda}_1, \ldots, \bar{\lambda}_{n-m})$ des reduzierten
Zustandsbeobachters
\begin{eqnarray}
    \dot{\hat{\xi}} &=& F \hat{\xi} + TBu + L y,\ t>0, \hat{\xi}(0)=\hat{\xi}_0\in\mathbb{R}^{n-m} \\
    \hat{x} &=& \Psi y + \Theta \hat{\xi}
\end{eqnarray}

der Ordnung $n-m$ bei $m$ Messgrößen $y$ mit
\begin{equation}
    \begin{pmatrix} \Psi & \Theta \end{pmatrix} =
    {\begin{pmatrix} C \\ T \end{pmatrix}}^{-1}
\end{equation}
die Eigenwerte $\bar{\lambda}_i$ und die zugehörigen Parametervektoren $l_i$. 

\section{Führungs- und Störbeobachter}
\subsection{Signalmodellbeobachter}
Mittels Signalbeobachter lassen sich modellgestützte Vorsteuerungen für das Führungs- und Störverhalten mit allg.
Signalformen entwerfen:
\begin{itemize}
    \item Allg. modellgestützte Vorsteuerung des Führungsverhaltens:
        \begin{eqnarray}
            \dot{\hat{v}}_w &=& S_w \hat{v}_w + L_w (w-Q_w \hat{v}_w) \\
            \dot{x}_s &=& A x_s + B u_s \\
            u_s &=& -K x_s + M_{v_w} \hat{v}_w
        \end{eqnarray}
    \item Allg. modellgestützte Vorsteuerung des Störverhaltens:
        \begin{eqnarray}
            \dot{\hat{v}}_d &=& S_d \hat{v}_d + L_d (d-Q_d \hat{v}_d) \\
            \dot{x}_s &=& A x_s + B u_s \\
            u_s &=& -K x_s + M_{v_d} \hat{v}_d
        \end{eqnarray}
\end{itemize}

\subsection{Klassischer Störbeobachter}
Problem: Störgrößen nicht direkt messbar.

Ansatz: Erweitertes System mit Stördynamik, dafür Zustandsbeobachter.

\subsubsection{Beobachtbarkeit des erweiterten Systems}
Das erweiterte System
\begin{eqnarray}
    \dot{x_e} &=& \begin{pmatrix}
            S_d & 0 \\
            G Q_d & A
        \end{pmatrix}
        x_e +
        \begin{pmatrix} 0 \\ B \end{pmatrix} u \\
    y &=& C_e x_e
\end{eqnarray}
ist beobachtbar, wenn
\begin{itemize}
    \item Minimales Störmodell: $(Q_d, S_d)$ beobachtbar
    \item Streckenzustandsrekonstruierbarkeit: $(C, A)$ beobachtbar
    \item Störmodellzustandsrekonstruierbarkeit:
        \begin{equation}
            \rank \begin{pmatrix} 
                    A-sI & G \\
                    C & 0
                \end{pmatrix}
            = n + q,\ \forall s \in \sigma(S_d)
        \end{equation}
        $(C, A, G)$ besitzt keine invarianten Nullstellen an Störmodelleigenwerten.
\end{itemize}

\subsection{Verallgemeinerter Störbeobachter (Dualer Davisonansatz)}
Wenn
\begin{itemize}
    \item $(A, B)$ steuerbar
    \item $(C, A)$ beobachtbar
    \item System besitzt keine invarianten Nullstellen an Störmodelleigenwerten:
        \begin{equation}
            \det \begin{pmatrix} 
                    A-sI & G \\
                    C & 0
                \end{pmatrix} \neq 0,\ \forall s \in \sigma(S)
        \end{equation}
\end{itemize}
erfüllt ist, dann existiert der verallgemeinerte Störbeobachter mit Störgrößenaufschaltung
\begin{eqnarray}
    \dot{\hat{v}} &=& S^* \hat{v} + L_v (y- C \hat{x}),\ \hat{v}(0)=\hat{v}_0 \in \mathbb{R}^{n_v} \\
    \dot{\hat{x}} &=& A \hat{x} + Bu + BP\hat{v} + L (y-C\hat{x}),\ \hat{x}(0)=\hat{x}_0 \in \mathbb{R}^n \\
    u&=& -K \hat{x} - P \hat{v}
\end{eqnarray}
zur exponentiellen Stabilisierung der resultierenden Regelung und stellt robuste
asymptotische Störkompensation sicher. Darin ist die Wahl von $p_i \in \mathbb{R}^{n_d}$
($i \in \{1, \ldots, p\}$) mit $(\T{p_i}, S)$ beobachtbar ein Freiheitsgrad.


