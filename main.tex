\documentclass[10pt]{report}

\usepackage{amsmath}
\usepackage{amsfonts}
\usepackage{amssymb}
\usepackage{enumitem}
\usepackage[utf8]{inputenc}
\usepackage[T1]{fontenc} 
\usepackage[ngerman]{babel}
\usepackage{commath}
\usepackage{xcolor}
\usepackage{booktabs}
\usepackage{float}
\usepackage{tikz-timing}
\usepackage{tikz}
\usepackage{multirow}
\usepackage{colortbl}
\usepackage{xstring}
\usepackage{circuitikz}
\usepackage{listings}
\usepackage{todonotes}
\usepackage[final]{pdfpages}
\usepackage{listings}
\usepackage{subcaption}
\usepackage{circuitikz}
\usepackage{varwidth}
\usepackage{algpseudocode}
\usepackage{trfsigns}
\usepackage{hyperref}
\usepackage[withpage]{acronym}
\usepackage{mathtools}
\usepackage{graphicx}
\usepackage{tcolorbox}


\usetikzlibrary{calc,shapes.multipart,chains,arrows}

\newcommand{\Prb}[1]{P(\text{#1})}
\newcommand{\CPr}[2]{P(\text{#1}|\text{#2})}
\DeclareMathOperator*{\dx}{\text{ d}x}
\DeclareMathOperator*{\argmax}{arg\,max}
\DeclareMathOperator*{\argmin}{arg\,min}
\DeclareMathOperator*{\sign}{sign}
\DeclareMathOperator*{\trace}{trace}
\DeclareMathOperator*{\diag}{diag}
\DeclareMathOperator*{\rank}{rank}
\newcommand{\T}[1]{{#1}^\text{T}}
\newcommand{\sysm}[0]{\dot{x} &=& Ax + Bu,\ t>0, x(0)=x_0\in \mathbb{R}^n}
\newcommand{\sys}[0]{\dot{x} = Ax + Bu,\ t>0, x(0)=x_0\in \mathbb{R}^n}
\newcommand{\sysom}[0]{y &=& C x,\ t \geq 0}
\newcommand{\syso}[0]{y = C x,\ t \geq 0}
\newcommand{\syo}[0]{\begin{eqnarray} \sysm \\ \sysom \end{eqnarray}}
\newcommand{\sy}[0]{\begin{equation} \sys\end{equation}}


\DeclarePairedDelimiter\ceil{\lceil}{\rceil}
\DeclarePairedDelimiter\floor{\lfloor}{\rfloor}

\tikzset{
    max width/.style args={#1}{
        execute at begin node={\begin{varwidth}{#1}},
        execute at end node={\end{varwidth}}
    }
}

\setlength\parindent{0pt}

\title{Systemtheorie -- Mehrgrößenregelung im Zustandsraum}
\author{Paul Nykiel}

\begin{document}
    \maketitle
    \pagebreak
    This page is intentionally left blank.
    \pagebreak
    \tableofcontents
    \pagebreak
    \chapter{Zustandsbeschreibung linearer Mehrgrößensysteme}
\section{Zustandsbeschreibung}
\subsection{Zustandsdarstellung eines nichtlinearen Systems}
Das Anfangswertproblem (AWP)
\begin{equation}
    \dot{x} = f(x, u), t > 0, x(0) = x_0 \in \mathbb{R}^n
\end{equation}
ist die zeitkontinuierliche Zustandsdarstellung $n$-ter Ordnung eines nichtlinearen
Systems für eine nichtlineare Vektorfunktion $f(x(t), u(t)): \mathbb{R}^n \times
\mathbb{R}^p \to \mathbb{R}^n$ mit dem Zustand $x(t) \in \mathbb{R}^n$ und der
Eingangsgröße $u(t) \in \mathbb{R}^p$ ($1 \leq p \leq n$), wenn es für den Anfangswert
$x(0) \in \mathbb{R}^n$ wohlgestellt ist.

\subsection{Wohlgestellheit nach Hadamard}
Es muss gelten:
\begin{itemize}
    \item Lösung des AWP existiert und ist eindeutig
    \item Lösung des AWP hängt stetig vom Anfangswert ab
\end{itemize}

Ansätze zur Überprüfung:
\begin{itemize}
    \item Zustandsdarstellungen die aus der Physik abgeleitet sind, sind in der Regel
        wohlgestellt
    \item Existenz- und Eindeutigkeitssätze aus der Theorie der Differentialgleichungen
\end{itemize}

\subsection{Ausgang}
\subsubsection{Definition}
In der Regelungstechnik sind nur manche Zustandsgrößen von Interesse, daher wird
eine Ausgangs-  oder Messgröße eingeführt:
\begin{equation}
    y = h(x,u)
\end{equation}
mit $y \in \mathbb{R}^m$. Da der Durchgriff nur in wenigen Systemen relevant ist
wird er oft weggelassen.

\subsubsection{Statisches System}
Besitzt ein System keinen Zustand, so wird es statisches System genannt:
\begin{equation}
    y = h(u)
\end{equation}

\subsubsection{Mehrgrößensysteme}
Gilt $p>1$ oder $m>1$, so spricht man von einem Mehrgrößensystem.

\subsection{Lineare Systeme}
Ein System:
\begin{eqnarray}
    \dot{x} &=& f(x, u),\ t>0, x(0)=x_0 \in \mathbb{R}^n \\
    y &=& h(x),\ t\geq 0
\end{eqnarray}
mit der Lösung $x(t) = \varphi(x_0, u(t), t)$ heißt linear, falls für alle Zeitpunkte
$t \geq 0$ und für alle Anfangszustände $x_0$ und Eingänge $u(t)$ für die
Ausgangsgröße $y(u, x_0, t) = h(\varphi(x_0, u(t), t))$ gilt:
\begin{eqnarray}
    y(\alpha_1 u_1 + \alpha_2 u_2, 0, t) &=& \alpha_1 y(u_1, 0, t) + \alpha_2 y(u_2, 0, t) \\
    y(0, \beta_1 x_{0,1} + \beta_2 x_{0,2}, t) &=& \beta_1 y(0, x_{0,1},t) + \beta_2 y(0, x_{0,2}, t) \\
    y(u, x_0, t) &=& y(0, x_0, t) + y(u, 0, t)
\end{eqnarray}
für $\alpha_1, \alpha_2, \beta_1, \beta_2 \in \mathbb{R}$ beliebig.

Hinweise:
\begin{itemize}
    \item Es wird keine Zustandslinearität, nur E/A Linearität gefordert
    \item Die Definition fordert keine Zeitinvarianz
\end{itemize}

\subsection{LTI Systeme}
Ein lineares Zeitinvariantes System kann durch
\begin{eqnarray}
    \dot{x} &=& Ax + Bu,\ t>0, x(0) = x_0 \in \mathbb{R}^n \\
    y &=& C x,\ t \geq 0 
\end{eqnarray}
beschrieben werden. Hierbei gilt: $A \in \mathbb{R}^{n \times n}, B \in \mathbb{R}^{n \times p}, C \in \mathbb{R}^{m \times n}$.

Es wird stets angenommen: $\rank{B} =p$ und $\rank{C}=m$.

\subsection{Ruhelage}
Die konstante Lösung $x_R(t) =x_R$ heißt Ruhelage des nichtlinearen Systems
\begin{equation}
    \dot{x} = f(x,u)
\end{equation}
wenn für $u(t) = u_R$ gilt:
\begin{equation}
    f(x_R, u_R) = 0
\end{equation}

\subsection{Jacobi-Linearisierung}
Eine Funktion $f: D \to \mathbb{R}^n$ sei auf dem Gebiet $D \subseteq \mathbb{R}^n$
zweimal stetig differenzierbar und die Punkte $x_0$ und $x_0 + \Delta x$ liegen mitsamt
ihrer Verbindungsstrecke in $D$. Dann gilt:
\begin{equation}
    f(x) = f(x_0) + \frac{\partial f}{\partial x}(x_0) (x-x_0) + \mathcal{O}^{[1+]}
        (x-x_0)
\end{equation}
hierbei erfüllt das Restglied:
\begin{equation}
    \lim_{x\to x_0} \frac{\mathcal{O}^{[1+]}(x-x_0)}{\abs{x-x_0}} = 0
\end{equation}

\section{Lösung der Zustandsgleichung}
\subsection{Transitionsmatrix}
Das unerregte System $\dot{x} = Ax$ wird gelöst durch:
\begin{equation}
    x(t) = \Phi(t) x_0
\end{equation}
mit der Transistionsmatrix
\begin{equation}
    \Phi(t) = e^{At}
\end{equation}

\subsection{Eigenschaften der Transitionsmatrix}
Die Transitionsmatrix $\Phi(t): \mathbb{R}^+ \to \mathbb{R}^{n \times n}$ besitzt folgende
Eigenschaften:
\begin{itemize}
    \item Anfangswert:
        \begin{equation}
            \Phi(0) = I
        \end{equation}
    \item Produkteigenschaft:
        \begin{equation}
            \Phi(t + \tau) = \Phi(t) \Phi(\tau)
        \end{equation}
    \item Invertierbarkeit:
        \begin{equation}
            \Phi^{-1}(t) = \Phi(-t) 
        \end{equation}
    \item Differenzierbarkeit:
        \begin{equation}
            \frac{\text{d}}{\text{d}t} \Phi(t) = A \Phi(t)
        \end{equation}
\end{itemize}
Damit bildet die Transitionsmatrix eine Gruppe.

\subsection{Lösung der Zustandsdifferentialgleichung}
Die ZDGL eine LTI-Systems:
\begin{equation}
    \dot{x} = Ax + Bu,\ t>0, x(0) = x_0 \in \mathbb{R}^n
\end{equation}
besitzt die eindeutige Lösung
\begin{equation}
    x(t) = e^{A t} x(0) + \int_0^t e^{A (t-\tau)} B u(\tau) \text{d}\tau
\end{equation}

\subsection{Darstellung über Eigenvektoren}
$w_i$ sind die Linkseigenvektoren, $v_i$ die Rechtseigenvektoren ($W=V^{-1}$) von $A$.

Dann gilt:
\begin{equation}
    x(t) = \sum_{i=1}^n v_i \T{w_i} e^{\lambda_i t} x(0)
        = \sum_{i=1}^n (\T{w_i} x(0)) e^{\lambda_i t} v_i
\end{equation}

Interpretation:
\begin{itemize}
    \item Eigenwerte legen die Form der Eigenschwingung fest
    \item Eigenwerte bestimmen Verteilung der Schwingung auf $x$
    \item Anfangswerte bestimmen die Anregung der Eigenbewegungen
\end{itemize}

\section{Zustandsstabilität}
\subsection{Asymptotische Stabilität}
Ein lineares System heißt asymptotisch stabil, wenn die Zustandstrajektorie $x(t)$ für
$t \to \infty$ gegen $x_R=0$ strebt, und zwar $\forall x_0 \in \mathbb{R}^n$.

\subsection{Eigenwertkriterium}
Ein lineares System ist genau dann asymptotisch stabil, falls $\Re(\lambda_i)<0,
\forall i \in \{1, \ldots, n\}$.

\subsection{Exponentielle Stabilität}
Ein lineares, asymptotisch stabiles System ist stets auch exponentiell stabil,
d.h.:
\begin{equation}
    \norm{x(t)} \leq M e^{-\alpha t} \norm{x_0},\ t \geq 0, \forall x(0) \in \mathbb{R}^n
\end{equation}
für $M\geq 1$ mit der Stabilitätsreserve $\alpha=-\max_{\lambda \in \sigma(A)}(\Re(\lambda))
$

\section{Steuerbarkeit}
\subsection{Definition}
Ein lineares System
\begin{equation}
    \dot{x} = Ax+Bu,\ t>0, x(0)=x_0 \in \mathbb{R}^n
\end{equation}
heißt steuerbar, wenn es in endlicher Zeit $T$ von einem beliebigen Anfangszustand $x(0)
\in \mathbb{R}^n$ durch eine Eingangsgröße $u(t), 0 \leq t \leq T$ in jeden beliebigen
Endzustand $x(T) \in \mathbb{R}^n$ überführt werden kann.

\subsection{Cayley-Hamilton}
Jede Matrix $A \in \mathbb{R}^{n \times n}$ genügt ihrer characteristischen Gleichung
\begin{equation}
    \det(\lambda I - A) = 0
\end{equation}
d.h. $A^n$ ist linear abhängig von $A^i, i \in \{0, \ldots, n-1\}$.

\subsection{Kalmansches Steuerbarkeitskriterium}
Das System
\begin{equation}
    \dot{x} = Ax + Bu,\ t>0, x(0)=x_0\in \mathbb{R}^n
\end{equation}
ist genau dann steuerbar, wenn die Steuerbarkeitsmatrix
\begin{equation}
    Q_s = \begin{pmatrix}
        B & AB & A^2 B & \ldots & A^{n-1} B
    \end{pmatrix}
\end{equation}
vollen Rang $n$ hat.

\section{Beobachtbarkeit}
\subsection{Definition}
Ein lineares System
\begin{eqnarray}
    \dot{x} &=& Ax + Bu,\ t>0, x(0)=x_0\in \mathbb{R}^n \\
    y &=& C x,\ t \geq 0
\end{eqnarray}
ist beobachtbar, wenn ein beliebiger Anfangszustand $x(0)$ aus Kenntnis des Ein- und
Ausgangs $u(t)$ und $y(t)$ auf einem Intervall $0\leq t \leq T$ in endlicher Zeit
bestimmt werden kann.

\subsection{Kalmansches Beobachtbarkeitskriterium}
Das System
\begin{eqnarray}
    \sysm \\
    \sysom \\
\end{eqnarray}
ist genau dann beobachtbar, wenn die Beobachtbarkeitsmatrix
\begin{equation}
    Q_B = \begin{pmatrix}
        C \\ C A \\ \vdots \\ C A^{n-1}
    \end{pmatrix}
\end{equation}
vollen Rang $n$ hat, d.h. $\rank{Q_B} = n$.

\section{Ein-/Ausgangsverhalten}
\subsection{Übertragungsmatrix}
Die Übertragungsmatrix des LTI Systems
\begin{eqnarray}
    \sysm \\
    \sysom
\end{eqnarray}
ist die analytische Fortsetzung von
\begin{equation}
    F(s) = C{(sI-A)}^{-1}B
\end{equation}

\subsection{Gewichtsmatrix}
Die Gewichtsmatrix des LTI-Systems
\syo
ist durch
\begin{equation}
    G(t) = C e^{At} B,\ t \geq 0
\end{equation}
gegeben.

\subsection{BIBO-Stabilität}
Das LTI-System
\syo
heißt BIBO-Stabil, wenn für beliebige beschränkte Eingangssignale, d.h.
\begin{equation}
    \norm{u(t)} \leq u_\text{max} < \infty,\ t \geq 0
\end{equation}
auch das Ausgangssignal beschränkt bleibt, d.h.:
\begin{equation}
    \norm{y(t)} \leq y_\text{max} < \infty,\ t \geq 0
\end{equation}

\subsection{Kriterium für BIBO-Stabilität}
Das LTI-System
\syo
ist BIBO-stabil genau dann, wenn alle Elemente $G_{ij}(t)$ der Gewichtsmatrix $G(t)$
absolut integrierbar sind, d.h.
\begin{equation}
    \int_0^\infty \abs{G_{ij}(\tau)} \text{d}\tau < \infty,\ 
    i \in \{1, \ldots, m\}, j \in \{1, \ldots, p\}
\end{equation}
oder die Übertragungsfunktion $F(s)$ keine Pole in der abgeschlossenen rechten komplexen
Halbebene besitzt.

Bemerkung: Aus Zustandstabilität folgt Übertragungsstabilität, nicht jedoch in die andere
Richtung.

    \chapter{Stabilisierung linearer Systeme im Zustandsraum}

    \chapter{Berücksichtigung von Führungs- und Störgrößen}
\section{Problemstellung}
Ausgangspunkte: LTI System mit zusätzlichem Störeingang
\begin{eqnarray}
    \dot{x} &=& Ax + Bu + Gd,\ t>0, x(0)=x_0 \in \mathbb{R}^n \\
    y &=& C x
\end{eqnarray}
Asymptotisches Folgen soll sichergestellt werden:
\begin{equation}
    \lim_{t \to \infty} e_y = \lim_{t \to \infty} y(t) - w(t) \stackrel{!}{=} 0 
\end{equation}

\section{Modellierung von Führungs- und Störgrößen}
Führungs- und Störgrößen als Lösung einer ODE darstellen:
\begin{eqnarray}
    \dot{v} &=& Sv,\ t>0, v(0)=v_0\in \mathbb{R}^{n_v} \\
    d &=& Pv,\ t \geq 0 \\
    w &=& Qv,\ t \geq 0
\end{eqnarray} 
mit $(P,S)$ und $(Q,S)$ beobachtbar. Resultat: Es muss nur der (unbekannte) Anfangszustand
$v_0$ Kompensiert werden, keine Dauerstörung, d.h. Zustandsrückführung ausreichend.


\section{Führungs- und Störgrößenaufschaltung}
Die Zustandsrückführung mit Führungs- und Störgrößenaufschaltung
\begin{equation}
    u = -Kx + (K \Pi + \Gamma) v
\end{equation}
erzielt asymptotisches Folgen, d.h.
\begin{equation}
    \lim_{t \to \infty} e_y(t) = 0,\ \forall x(0) \in \mathbb{R}^n, \forall v(0) \in
        \mathbb{R}^{n_v}
\end{equation}
wenn
\begin{itemize}
    \item $A-BK$ ist Hurwitz Matrix
    \item $\Pi \in \mathbb{R}^{n\times n_v}$ und $\Gamma \in \mathbb{R}^{p \times n_v}$ sind Lösung der \glqq{}regulator equations\grqq{}:
        \begin{eqnarray}
            \Pi S - A \Pi &=& GP + B \Gamma \\
            C \Pi &=& Q
        \end{eqnarray}
\end{itemize}

\subsection{Starke Übertragungsblockierung}
Das Eingangssignal $u(t) = u_0 e^{\eta t}, \eta \notin \sigma(A)$, wird für $x(0)=c_p$
nicht zum Ausgang übertragen, bzw. stark übertragungsblockiert, falls:
\begin{equation}
    \begin{pmatrix}
        A - \eta I & B \\
        C & 0
    \end{pmatrix}
    \begin{pmatrix} c_p \\ u_0 \end{pmatrix} = 0
\end{equation}

\subsection{Invariante Nullstellen}
Die Lösungen $\eta$ von
\begin{equation}
    \det P(s) = 0
\end{equation}
mit der Rosenbrock-Matrix
\begin{equation}
    P(s) = 
    \begin{pmatrix}
        A - s I & B \\
        C & 0
    \end{pmatrix}
\end{equation}

heißen invariante Nullstellen der Strecke $(C, A, B)$ mit der Zustandsnullstellenrichtung
$c_p$ und der Eingangsnullstellenrichtung $u_0$.

Hinweise:
\begin{itemize}
    \item $\det F(s)=0$ kann nicht zur Berechnung der invarianten Nullstellen genutzt werden
    \item Invariante Nullstellen sind Systemgrößen des Zustandsmodells, nicht des
        Übertragungsverhaltens.
    \item Können nicht durch Zustandsrückführung verschoben werden
\end{itemize}

\subsection{Lösbarkeit der Regulator Equations}
Die \glqq{}regulator equations\grqq{} haben für $\dim y = \dim u$ eine Eindeutige Lösung
$(\Pi, \Gamma)$, wenn keine invarianten Nullstellen von $(C, A, B)$ mit einem
Eigenwert von $S$ übereinstimmt.

\subsection{Diskussion}
Vorteile:
\begin{itemize}
    \item Regelgröße $y$ muss nicht gemessen werden
    \item Keine Verringerung der Stabilitätsreserve der Regelung
\end{itemize}

Nachteile:
\begin{itemize}
    \item Asymptotische Störkompensation vom Eingriffsort abhängig
    \item Nicht robust (bei Modellunbestimmtheiten)
\end{itemize}
 
\section{Internes-Modell-Prinzip}
Annahme: $w, d$ wieder über Signalmodell $S$, $P$, $Q$ können aber unbekannt sein.

Signalmodell:
\begin{equation}
    \dot{\bar{v}}_i = S \bar{v}_i + b_{y_i} (y_i - w_i)
\end{equation}
Anregung mit Eigenschwingungen, also Aufklingen von $\bar{v}_i$. Stabilisierung
des erweiterten Systems (System + Signalmodell).

\subsection{Davisonansatz}
Zusammenfassung der Signalmodelle (ohne Führungsgröße):
\begin{equation}
    \dot{\bar{v}} = S^* \bar{v} + B_y y
\end{equation}
mit ($b_y$ jeweils Steuerbar)
\begin{eqnarray}
    \bar{v} &=& \begin{pmatrix}
            \bar{v}_1 \\ \vdots \\ \bar{v}_p
        \end{pmatrix} \\
    S^* &=& \begin{pmatrix}
            S \\
             & \ddots \\
             & & S
        \end{pmatrix} \\
    B_y &=& \begin{pmatrix}
            b_{y_1} \\
             & \ddots \\
             & & b_{y_p}
        \end{pmatrix}
\end{eqnarray}

erweitertes System:
\begin{eqnarray}
    \begin{pmatrix} \dot{\bar{v}} \\ \dot{x} \end{pmatrix} &=&
        \begin{pmatrix} S^* & B_y C \\ 0 & A \end{pmatrix}
        \begin{pmatrix} \bar{v} \\ x \end{pmatrix} +
        \begin{pmatrix} 0 \\ B \end{pmatrix} u \\
    y &=& \begin{pmatrix} 0 & C \end{pmatrix}
        \begin{pmatrix} \bar{v} \\ x \end{pmatrix}
\end{eqnarray}

\subsection{Stabilisierbarkeit des erweiterten Systems}
Das erweiterte System
\begin{eqnarray}
    \begin{pmatrix} \dot{\bar{v}} \\ \dot{x} \end{pmatrix} &=&
        \begin{pmatrix} S^* & B_y C \\ 0 & A \end{pmatrix}
        \begin{pmatrix} \bar{v} \\ x \end{pmatrix} +
        \begin{pmatrix} 0 \\ B \end{pmatrix} u \\
    y &=& \begin{pmatrix} 0 & C \end{pmatrix}
        \begin{pmatrix} \bar{v} \\ x \end{pmatrix}
\end{eqnarray}
ist steuerbar, wenn
\begin{itemize}
    \item $(A, B)$ steuerbar
    \item $(S, b_{y_i}$ ($i \in \{1, \ldots, p\}$) steuerbar
    \item Keine invariante Nullstelle von $(C, A, B)$ stimmt mit einem
        Eigenwert von $S$ überein.
\end{itemize}

Ergebnis: Dynamische Zustandsrückführung für robustes asymptotisches Folgen:
\begin{eqnarray}
    \dot{\bar{v}} &=& S^* \bar{v} + B_y (y-w) \\
    u &=& -K_{\bar{v}} \bar{v} - K_x x
\end{eqnarray}

\subsection{Diskussion}
Vorteile:
\begin{itemize}
    \item Robust
    \item Unabhängig vom Eingriffsort
\end{itemize}

Nachteile:
\begin{itemize}
    \item Reglerdifferenz muss messbar sein
    \item Signalmodell verringert Stabilitätsreserve
    \item Regler hat Eigenwerte auf der Imaginärachse (Windup Vermeidung)
\end{itemize}

    \chapter{Steuerungsentwurf im Zustandsraum}
\section{Modellgestütze Vorsteuerung}
\subsection{Führungsgrößenaufschaltung}
\subsubsection{Definition Exaktes Folge}
Die Regelgröße folgt dem Sollverlauf $y_s$ exakt, wenn
\begin{equation}
    y(t) = y_s(t),\ \forall t \geq 0
\end{equation}
bzw. $y \equiv y_s$.

\subsubsection{Definition Asymptotisches Folgen}
Die Regelgröße folgt der Führungsgröße $w$ asymptotisch, wenn
\begin{equation}
    \lim_{t \to \infty} y(t) - w(t) = 0
\end{equation}
dabei soll $y$ den Verlauf von $w$ schnell und gut gedämpft annehmen.

Ansatz: Vorgabe des Folgeverhaltens durch Festlegung des Führungsübertragungsverhaltens:
\begin{equation}
    y_s(s) = F_w(s) w(s) \stackrel{!}{=} y(s)
\end{equation}

\subsubsection{Führungsgrößenaufschaltung}
Ansatz:
\begin{equation}
    u = -K x + M w
\end{equation}
Bestimmung $K$ über Eigenwertvorgabe oder Ricatti Regler.

Vorfilter $M$ für stationäre Genauigkeit:
\begin{equation}
    M = {\left(C{(-A+BK)}^{-1}B\right)}^{-1}
\end{equation}

Voraussetzungen für Existenz:
\begin{itemize}
    \item Kein Regelungseigenwert bei Null
    \item Keine invariante Nullstelle der Strecke bei $0$
\end{itemize}

\subsection{Ein-/Ausgangsentkopplung}
\subsubsection{Definition Differenzgrad}
Die Ausgangsgröße $y_{s,i}$ hat Differenzgrad $r_i$, wenn:
\begin{eqnarray}
    \T{c_i} A^k B &=& \T{0},\ k \in \{0, 1, \ldots, r_i-2\} \\ 
    \T{c_i} A^{r_i-1} B &\neq& \T{0}
\end{eqnarray}
Die Summe $r = \sum_i r_i \leq n$ heißt Differenzgrad des Systems.

\subsubsection{Entkoppelbarkeitskriterium}
Ein LTI-System ist ein-/ausgangsentkoppelbar genau dann wenn $\det D^* \neq 0$ mit:
\begin{equation}
    D^* = \begin{pmatrix}
            \T{c_1} A^{r_1-1} B \\
            \vdots \\
            \T{c_p} A^{r_p-1} B \\
        \end{pmatrix}
\end{equation}

dann können die $p$ Integrierketten entkoppelt werden:
\begin{equation}
    u_s = {(D^*)}^{-1} (-A^* x_s + \bar{u}_s)
\end{equation}

\subsubsection{Entkopplungsregler nach Falb-Wolovich}
Der Entkopplungsregler 
\begin{equation}
    u_s = - {(D^*)}^{-1} \begin{pmatrix}
            \T{k_1^*} \\ \vdots \\ \T{k_m^*}
        \end{pmatrix} +
        {(D^*)}^{-1} \begin{pmatrix}
            \tilde{a}_0^1 \\
             & \ddots \\
             & & \tilde{a}_0^p
        \end{pmatrix} w
\end{equation}
mit
\begin{equation}
    D^* = \begin{pmatrix}
            \T{c_1} A^{r_1-1} B \\
            \vdots \\
            \T{c_p} A^{r_p-1} B \\
        \end{pmatrix}
\end{equation}
und
\begin{equation}
    \T{k^*_i} = \T{c_i} \left(A^{r_i} + \sum_{j=0}^{r_i-1} \tilde{a}_j^i A^j \right)
\end{equation}

bewirkt die Ein-/Ausgangsentkopplung des vorgesteuerten Führungsverhaltens, d.h.:
\begin{equation}
    F_w(s) = \begin{pmatrix}
        \frac{\tilde{a}_0^1}{s^{r_1} + \tilde{a}^1_{r_1-1} s^{r_1-1} + \ldots + \tilde{a}_0^1} \\
        & \ddots \\
        & & \frac{\tilde{a}_0^p}{s^{r_p} + \tilde{a}^p_{r_p-1} s^{r_p-1} + \ldots + \tilde{a}_0^p} 
    \end{pmatrix}
\end{equation}

Stabilität:

$n-r$ Eigenwerte $\tilde{\lambda}_i$ ($i\in\{r+1, \ldots, n\}$) werden mit invarianten
Nullstellen kompensiert.

\subsubsection{Hinreichendes Stabilitätskriterium für stabile Entkopplung}
Die Entkopplungsregelung ist exponentiell stabil, wenn die Strecke minimalphasig ist,
d.h. wenn für alle invarianten Nullstellen $\eta_i$ ($i\in \{1, \ldots, n-r\}$)
\begin{equation}
    \Re(\eta_i) < 0
\end{equation}
gilt.

\subsection{Vorsteuerung des Störverhaltens}

\section{Inversionsbasierte Ausgangsfolge}
\subsection{Ein-/Ausgangsnormalform}

\subsection{Ausgangsfolge}

\section{Flachheitsbasierter Arbeitspunktwechsel}
\subsection{Arbeitspunktwechsel}

\subsection{Flachheit linearer Systeme}

\subsection{Bestimmung flacher Ausgänge}

\subsection{Flachheitsbasierter Steuerungsentwurf}

\subsection{Arbeitspunktwechsel für die Regelgröße}

    \chapter{Lineare Beobachter}

    \chapter{Zwei-Freiheitsgrade-Regelung}
\section{Struktur der Folgeregelung}
\subsubsection{Online Sollvorgabe}
Komponenten:
\begin{itemize}
    \item Modellgestützte $w$-Aufschaltung
    \item Modellgestützte $d$-Aufschaltung
    \item Ausgangsfolgeregler
\end{itemize}

Vorteile: Führungsverhalten, und Störverhalten bezüglich jeweils $d$ und $r$ kann
separat ausgelegt werden.

\subsubsection{Offline Sollvorgabe}
Komponenten
\begin{itemize}
    \item Streckeninverse für Steuerung
    \item Ausgangsfolgeregler
\end{itemize}

\section{Ausgangsfolgeregler}
\subsection{Beobachterbasierte Zustandsrückführung}
Entwurf der beobachterbasierten Ausgangsrückführung:
\begin{enumerate}
    \item Entwurf einer Zustandsrückführung
        \begin{equation}
            u_R = -K e_x
        \end{equation}
    \item Entwurf eines reduzierten Zustandsbeobachters
    \item Implementierung des Stellgesetz auf basis Beobachterzustände
    \item Resultiert in dynamische Ausgangsrückführung:
        \begin{eqnarray}
            \begin{pmatrix}
                \dot{\hat{e_\xi}} \\ \dot{\bar{v}}
            \end{pmatrix}
            &=&
            \begin{pmatrix}
                F - TBR_\xi & - T B R_v \\
                0 & S^*
            \end{pmatrix}
            \begin{pmatrix}
                \hat{e_\xi} \\ \bar{v}
            \end{pmatrix}
            +
            \begin{pmatrix}
                L-TBR_y \\ B_{e_y}
            \end{pmatrix}
            e_y \\
            u_r &=& -R_y e_y - R_\xi \hat{e_\xi} - R_v \bar{v}
        \end{eqnarray}
\end{enumerate}

\subsection{Beobachterbasierte Ausgangsrückführung}
Ansatz: Stellgesetz auf basis des Reduzierten Beobachters kann weiter
reduziert werden wenn Dimension des reduzierten Beobachters ($n_0$) weiter reduziert wird.

\subsubsection{Ausgangsreglerformel}
Die Matrix
\begin{equation}
    R(\tilde{\lambda}_i, p_i, \bar{\lambda}_j, l_j) = -
        \begin{pmatrix} p_1 & \ldots & p_{m+n_0} \end{pmatrix}
        {\begin{pmatrix} G(\tilde{\lambda}_1) p_1 & \ldots &
        G(\tilde{\lambda}_{m+n_0})p_{m+n_0} \end{pmatrix}}^{-1}
\end{equation}
mit
\begin{equation}
    G(s) = \begin{pmatrix}
            F(s) \\
            \frac{\T{l_1}}{s-\bar{\lambda}_1} \left(F(s) - F(\bar{\lambda}_1)\right) \\
            \vdots \\
            \frac{\T{l_{n_0}}}{s-\bar{\lambda}_{n_0}} \left(F(s) - F(\bar{\lambda}_{n_0})\right)
        \end{pmatrix}
\end{equation}
für $\tilde{\lambda}_i \notin \left(\sigma(A) \cup \sigma(\Lambda)\right)$ und
$\bar{\lambda}_i \notin \sigma(A)$ verleiht dem geregelten Fehlersystem:
\begin{equation}
    \dot{e_x} = (A-BRH) e_x
\end{equation}
die Eigenwerte $\tilde{\lambda}_i$ ($i \in \{1, \ldots, m+n_0\}$) und die zugehörigen 
Parametervektoren $p_i$. Darin stellen die Beobachterreigenwerte $\bar{\lambda}_i$ und
ihre Parametervektoren $\T{l_i}$ weitere FHG dar.

\subsubsection{Reduzierter Ausgangsfolgeregler}
Bei Verwendung eines $e_\xi$-Beobachters der Ordnung $n_0 < n-m$ können mit dem 
Ausgangsfolgeregler $m+n_0<n$ Eigenwerte direkt vorgegeben werden, während die restlichen
$n-(m+n_0)$ Eigenwerte durch eine Parameteroptimierung beeinflussbar sind.

\subsection{Separationsprinzip}
Die Eigenwerte des mittels einer beobachterbasierten Zustandsrückführung oder
Ausgangsrückführung geregelten Fehlersystems sind die Nullstellen von:
\begin{equation}
    \det (\lambda I - \tilde{A}) = \det(\lambda I - F) \det(\lambda I - A + BRH)
\end{equation}
also die Eigenwerte des Beobachters und der Fehlerzustandsrückführung, damit bleiben
die Dynamikvorgaben des getrennten Entwurfs im geschlossenen Regelkreis erhalten.
Folglich existiert ein stabilisierender Ausgangsregler, wenn das Fehlersystem (bzw.
die Strecke) steuer- und beobachtbar ist.

\section{Berücksichtigung von Stellsignalbegrenzungen}
\subsection{Ausgangsfolgeregler mit klassischem Davisonansatz}
\subsubsection{Beobachtertechnik}
Durch Einspeisung des begrenzten Stellsignals in den Beobachter wird die Rückführung
über den Eingang im Begrenzungsfall unterbrochen. Damit tritt bei Begrenzungen nur die
stabile Beobachterdynamik im offenen Kreis auf und das Beobachter-Windup wird vermieden.

\vspace{.5cm}

Hinweis: Stellsignalbegrenzung kann auch explizit vor System gehängt werden, damit
wird Stellsignalbegrenzung des Systems nie erreicht.

\vspace{.5cm}

Vermeidung des Windups der Zustände $\bar{v}$ der angeregten Störmodelle:

Einführung der Begrenzungsgröße:
\begin{equation}
    \Delta u  = u - u_b
\end{equation}
und zusätzliche Rückführung im Störmodell:
\begin{equation}
    \dot{\bar{v}} = S^* \bar{v} + B_{e_y} e_y + B_{\Delta u} \Delta u
\end{equation}
dafür muss $S^*-B_{\Delta u}R_v$ Hurwitz Matrix sein, das ist möglich falls $(R_v, S*)$
beobachtbar.

\subsection{Ausgangsfolgeregler mit dualem Davisonansatz}
Beobachtertechnik ausreichend.


    \appendix
\end{document}
